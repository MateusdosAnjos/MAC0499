%!TeX root=../tese.tex
%("dica" para o editor de texto: este arquivo é parte de um documento maior)
% para saber mais: https://tex.stackexchange.com/q/78101/183146
%% ------------------------------------------------------------------------- %%
\chapter{O Jogo}
\label{cap:introducao}

\section{Objetivo Dentro do Jogo}
\label{sec:consideracoes_preliminares}

Para que seja mais fácil entender o projeto como um todo, esta seção explicará
o objetivo que o jogador deve alcançar ao jogar 
\textit{Phoenix Rising}. A imagem abaixo exemplifica um nível do jogo.

\begin{figure}[H]
    \includegraphics[width=\textwidth]{../figuras/exemplo_nivel.png}
    \caption{Exemplo de um nível}
\end{figure}

O jogador irá receber dados que serão mostrados dentro do retângulo azul, 
posicionado no lado esquerdo da tela. O exemplo abaixo mostra um nível que
fornece  ao jogador os números 7 e 41 como dados iniciais.

\begin{figure}[H]
    \includegraphics[scale=0.3]{../figuras/entrada.png}
    \caption{Exemplo de entrada}
\end{figure}

O objetivo do jogador é conseguir reproduzir o que está no retângulo verde, 
posicionado no lado direito da tela,
chamado "Resposta Esperada". No exemplo abaixo o nível exige que o jogador
reproduza os números 8 e 42. Após o programa do jogador ser executado a saída 
que ele obteve aparecerá abaixo de "Sua Resposta" \ que está no retângulo vermelho
(fig(a)) e caso o resultado em "Sua Resposta" \ for igual ao que está em 
"Resposta Esperada" o retângulo também ficará verde (fig (b)).

\begin{figure}[H]
    \centering
    \begin{minipage}{.4\textwidth}
      \centering
      \includegraphics[scale=0.3]{../figuras/saida.png}
      \subcaption{Objetivo não alcançado}
    \end{minipage}%
    \begin{minipage}{.4\textwidth}
      \centering
      \includegraphics[scale=0.3]{../figuras/objetivo_alcancado.png}
      \subcaption{Objetivo alcançado}
    \end{minipage}
    \caption{Objetivos}
\end{figure}

O jogador deve cumprir o objetivo dentro do tempo limite para acumular pontos.
Este tempo é mostrado constantemente na tela e sempre iniciará com 120 segundos
restantes.

\begin{figure}[H]
    \includegraphics[scale=0.3]{../figuras/tempo_restante.png}
    \caption{Tempo restante no início do nível}
\end{figure}

Ao concluir o nível o jogador ganhará pontos iguais ao valor de tempo que 
restava ao apertar o botão "rodar!", obviamente os pontos só serão obtidos caso
o objetivo seja alcançado. Este fato está relacionado com a duração do processo 
visual que o jogo possui e será explicado mais adiante, por hora deve-se 
entender que existe um sistema de pontuação.

\begin{figure}[H]
    \includegraphics[scale=0.3]{../figuras/pontuacao.png}
    \caption{Pontuação obtida}
\end{figure}

Note que, se o jogador tinha zero pontos quando concluiu este nível, o programa
criado que alcançou o objetivo foi executado quando ainda havia 101 segundos
restantes, porém a animação continuou executando, consequentemente o tempo
continuou correndo, para que o usuário
pudesse visualizar o que está acontecendo durante a execução do programa criado
e aprender como funciona cada comando.

Portanto o objetivo final de \textit{Phoenix Rising} é completar o maior número
de níveis no menor tempo possível para maximizar o somatório de pontos. Para 
isso o jogador deve aprender a mecânica de jogo, o que  e como cada comando
executa sua instrução e como montar o quebra cabeça dos diferentes níveis.

\section{Elementos do Jogo}
\label{sec:consideracoes_preliminares}

Agora que o objetivo do jogo foi explicitado, deve-se entender quais elementos 
estão envolvidos para que o jogador possa concluir o desafio.

\subsection{Entrada e Saída}

Estes termos são recorrentes na computação e geralmente são chamados de
\textit{Intput}\footnote{Dados fornecidos para o sistema processar} e 
\textit{Output}\footnote{Dados que o sistema gera após o processamento} 
respectivamente. Neste jogo a entrada e saída
estão delimitadas pelos retângulos coloridos e servem para mostrar para o 
jogador o que ele receberá para processar e o que ele deve produzir com o código
gerado.

\begin{figure}[H]
    \centering
    \begin{minipage}{.4\textwidth}
      \centering
      \includegraphics[scale=0.3]{../figuras/exemplo_entrada.png}
      \subcaption{Exemplo de Entrada}
    \end{minipage}%
    \begin{minipage}{.4\textwidth}
      \centering
      \includegraphics[scale=0.3]{../figuras/exemplo_saida.png}
      \subcaption{Exemplo de Saída}
    \end{minipage}
    \caption{Entrada e Saída}
\end{figure}

\subsection{Espaço de Ação}

Os espaços de ação são as áreas móveis do jogo que permitem montar o 
quebra-cabeças e que recebem os comandos que executarão as ações de 
processamento dos dados de entrada. Portanto este é um dos principais itens do 
jogo.

\begin{figure}[H]
    \includegraphics[width=\textwidth]{../figuras/espaco_acao.png}
    \caption{Um espaço de ação}
\end{figure}
\newpage
O jogador pode:
\begin{itemize}
    \item[$\bullet$]
        Movimentar o espaço de ação ao clicar no ícone de arrastar.
        \begin{figure}[H]
            \centering
            \begin{minipage}{.4\textwidth}
              \centering
              \includegraphics[scale=0.1]{../figuras/open_hand_icon.png}
              \subcaption{Ícone de arrastar não pressionado}
            \end{minipage}%
            \begin{minipage}{.4\textwidth}
              \centering
              \includegraphics[scale=0.1]{../figuras/closed_hand_icon.png}
              \subcaption{Ícone de arrastar pressionado}
            \end{minipage}
            \caption{Ícones de arrastar}
        \end{figure}
    \item[$\bullet$]
        Modificar as conexões de entrada e saída ao clicar nos ícones de troca de 
        conexões.
        \begin{figure}[H]
            \includegraphics[scale=0.4]{../figuras/change_button.png}
            \caption{Ícone de trocas de conexões}
        \end{figure}
        \begin{figure}[H]
            \centering
            \begin{subfigure}{0.32\textwidth}
                \centering
                \includegraphics[width=1\textwidth]{../figuras/conexao_ex1.png}
            \end{subfigure}
            \begin{subfigure}{0.32\textwidth}
                \centering
                \includegraphics[width=1\textwidth]{../figuras/conexao_ex2.png}
            \end{subfigure}
            \begin{subfigure}{0.32\textwidth}
                \centering
                \includegraphics[width=1\textwidth]{../figuras/conexao_ex3.png}
            \end{subfigure}    
            \caption{Exemplos de conexões alteradas}
        \end{figure}
    \item[$\bullet$]
        Verificar qual a posição do espaço de ação na sequência de 
        operações.
        \begin{figure}[H]
            \includegraphics[scale=0.5]{../figuras/numero_posicao.png}
            \caption{Posição na ordem de ações (1)}
        \end{figure}
    \item[$\bullet$]
        Preencher os \textit{argumentos}\footnote{É um valor, proveniente 
        de uma variável ou de uma expressão mais complexa, que pode ser passado para 
        um comando (sub-rotina). Um comando utiliza os valores atribuídos aos 
        parâmetros para alterar o seu comportamento em tempo de execução.}
        necessários para os diferentes comandos
        \begin{figure}[H]
            \includegraphics[scale=0.6]{../figuras/argumentos.png}
            \caption{Espaço para Argumentos}
        \end{figure}
    \item[$\bullet$]
        Posicionar o comando a ser utilizado no respectivo espaço de ação.
        \begin{figure}[H]
            \includegraphics[scale=0.6]{../figuras/espaco_com_comando.png}
            \caption{Espaço de Ação com Comando \textit{Print}}
        \end{figure}
\end{itemize}

Note que, dentro do jogo, um espaço de ação não pode ser movimentado caso um 
comando esteja posicionado, isso faz com que o jogador tenha que completar o
quebra cabeças antes de pensar quais comandos serão utilizados.

\subsection{Inventário e Comandos}

Um comando é uma ação que processa o dado de uma forma específica de acordo com
os argumentos que recebe, portanto todo comando possui nome e uma função.
Os comandos do jogo estão em vermelho na imagem abaixo.

\begin{figure}[H]
    \includegraphics[scale=0.5]{../figuras/inventario_comandos.png}
    \caption{Inventário com Comandos}
\end{figure}

Os três primeiros comandos na imagem são de soma, subtração e multiplicação
respectivamente. Estes comandos executam as operações básicas como conhecemos e
apenas a operação de soma funciona como operador de concatenação caso o dado a
ser processado seja uma string.

O quarto comando da sequência, chamado \textit{Print}, escreve na saída 
"Sua Resposta" o valor do \textit{Input} ou de uma variável do programa,
dependendo de qual argumento passado.

O quinto comando da sequência, chamado \textit{Pass}, serve apenas para conectar
o sistema sem executar nenhum processamento dos dados.
O sexto comando da sequência, chamado \textit{If/Else}, serve como controle de 
fluxo do programa, ou seja, o comando recebe como argumento uma expressão, 
nomeada condição, e durante a execução o comando \textit{If/Else} avalia se 
tal condição é verdadeira ou falsa, executando o ramo de ação referente ao 
resultado da avaliação.

\begin{figure}[H]
    \centering
    \begin{subfigure}{0.48\textwidth}
        \centering
        \includegraphics[width=1\textwidth]{../figuras/avaliacao_if.png}
        \caption{Avaliação da condição}
    \end{subfigure}
    \begin{subfigure}{0.48\textwidth}
        \centering
        \includegraphics[width=1\textwidth]{../figuras/caminho_if.png}
        \caption{Caminho referente a avaliação}
    \end{subfigure}  
    \caption{Processo visual para If}
\end{figure}

\begin{figure}[H]
    \centering
    \begin{subfigure}{0.48\textwidth}
        \centering
        \includegraphics[width=1\textwidth]{../figuras/avaliacao_else.png}
        \caption{Avaliação da condição}
    \end{subfigure}
    \begin{subfigure}{0.48\textwidth}
        \centering
        \includegraphics[width=1\textwidth]{../figuras/caminho_else.png}
        \caption{Caminho referente a avaliação}
    \end{subfigure}  
    \caption{Processo visual para Else}
\end{figure}

Nos exemplos acima o input segue o caminho de cima caso ele seja maior que zero,
caso contrário seguirá o caminho de baixo. Para o sistema ter esse comportamento
basta passar como argumento para o comando \textit{If/Else} '' > 0 ''.

O sétimo e oitavo comandos, chamados \textit{A} e \textit{B} respectivamente, 
são variáveis e podem armazenar informações do programa para serem utilizadas
posteriormente. Além disso o jogador pode acompanhar os valores de \textit{A} e
\textit{B} durante a execução do programa olhando para a região de 
\textit{Valores das Variáveis} localizada no canto superior direito da tela.


\begin{figure}[H]
    \includegraphics[scale=0.8]{../figuras/valores_variaveis.png}
    \caption{Valores das Variáveis}
\end{figure}

Conforme a execução do programa estes valores serão modificados, veja na 
imagem abaixo.

\begin{figure}[H]
    \centering
    \begin{subfigure}{0.48\textwidth}
        \centering
        \includegraphics[width=0.6\textwidth]{../figuras/atribuicao_variavel.png}
        \caption{Atribuição do valor 42 para variável \textit{A}}
    \end{subfigure}
    \begin{subfigure}{0.48\textwidth}
        \centering
        \includegraphics[width=1\textwidth]{../figuras/modific_tabela_variavel.png}
        \caption{Mudança na região de \textit{Valores das Variáveis}}
    \end{subfigure}  
    \caption{Processo visual para mudança nas variáveis}
\end{figure}

Note também que o \textit{Input} era vazio, portanto valores de variáveis podem
ser valores fixos atribuidos pelo jogador, como o exemplo acima mostra, ou podem
ser dinâmicos, ou seja, o jogador pode armazenar em uma variável o valor 
corrente do \textit{Input}.

O nono comando, chamado \textit{Error}, não deve ser utilizado, pois só aparece
caso haja algum erro na obtenção dos comandos anteriores. Por conta disso
o comando de erro não possui comportamento algum e, se tudo der certo durante
o jogo, este comando não aparecerá em nenhum momento.

\subsection{Setas de Entrada e Saída}

Estas setas são as conexões iniciais com a qual o jogador deve se preocupar. O programa
do jogador irá receber os valores disponibilizados a serem processados a partir
da \textit{Seta de Entrada}, portanto a primeira conexão que deve
ser feita é entre um espaço de ação e esta seta. Já a \textit{Seta de Saída}
será a última conexão que o jogador terá que fazer, pois o programa só estará 
apto a ser executado quando existir uma conexão entre a duas setas.

\begin{figure}[H]
    \centering
    \begin{subfigure}{0.48\textwidth}
        \centering
        \includegraphics[width=0.6\textwidth]{../figuras/seta_entrada.png}
        \caption{Seta de onde saem dados de Entrada}
    \end{subfigure}
    \begin{subfigure}{0.48\textwidth}
        \centering
        \includegraphics[width=0.6\textwidth]{../figuras/seta_saida.png}
        \caption{Seta que recebe dados de Saída}
    \end{subfigure}  
    \caption{Setas de Entrada e Saída}
\end{figure}

\subsection{Botões}

Na tela de jogo constam os seguintes botões cujo comportamento está especificado
ao lado:

\begin{itemize}
    \item[$\bullet$]
        \textbf{Tela Cheia} - Coloca o jogo em tela cheia.
    \item[$\bullet$] 
        \textbf{Menu Principal} - Retorna ao menu principal
    \item[$\bullet$] 
        \textbf{Velocidade da Animação} - Modifica a velocidade da animação de execução
        do programa.
    \item[$\bullet$]
        \textbf{Rodar!} - Inicia a execução do programa criado caso o sistema esteja
        conectado e não haja erro na utilização dos comandos.
    \item[$\bullet$]
        \textbf{Reiniciar Nível} - Recomeça o nível atual.
    \item[$\bullet$]
        \textbf{Próximo Nível} - Inicia o nível subsequente do nível atual.
\end{itemize}

\section{Forma de Jogar}
\label{sec:consideracoes_preliminares}

Para conseguir completar o objetivo o jogo \textit{Phoenix Rising} funciona
da seguinte forma:

O jogador deve resolver o quebra cabeças conectando os blocos da forma correta
até que a Seta de Entrada esteja conectada com a Seta de Saída.

\begin{figure}[H]
    \includegraphics[width=\textwidth]{../figuras/jogo_nao_conectado.png}
    \caption{Jogo não conectado}
\end{figure}

\begin{figure}[H]
    \includegraphics[width=\textwidth]{../figuras/jogo_conectado.png}
    \caption{Jogo conectado}
\end{figure}

Desta forma o jogador deve compreender que, para criar um programa, é necessário
pensar sobre a estrutura que o código terá antes de começar a utilizar os 
comandos, pois tentar criar um código apenas inserindo comandos sem pensar
previamente em uma estrutura base leva a códigos confusos e que muitas vezes
não funcionam corretamente. É claro que para sistemas maiores as reestruturações
do modelo ocorrem com certa frequência, porém o objetivo deste jogo é apenas
introduzir os conceitos básicos de programação.

Após ter o sistema conectado, o jogador deve utilizar os comandos que são
disponibilizados no inventário, posicionados no canto inferior esquerdo da tela 
de jogo.

\begin{figure}[H]
    \includegraphics[scale=0.5]{../figuras/exemplo_comandos.png}
    \caption{Exemplo de comandos disponíveis}
\end{figure}

Depois de posicionar os comandos, o jogador deve preencher os argumentos que 
cada comando recebe e então o sistema estará pronto para ser executado.

\begin{figure}[H]
    \includegraphics[width=\textwidth]{../figuras/preenchendo_argumentos.png}
    \caption{Jogador preenchendo os argumentos}
\end{figure}

Agora o sistema está pronto para ser executado.

\begin{figure}[H]
    \includegraphics[width=100mm, height=80mm]{../figuras/sistema_pronto.png}
    \caption{Sistema pronto para execu\c{c}\~{a}o}
\end{figure}

Para iniciar o processamento dos dados de entrada, ou seja, rodar o programa,
basta o jogador clicar no botão \textit{rodar!} e ficar atento à animação.
No exemplo abaixo a primeira entrada era o número 7 e está sinalizada na 
animação pelo nome \textit{Input:}.

\begin{figure}[H]
    \includegraphics[width=\textwidth]{../figuras/inicio_da_execucao.png}
    \caption{Início da execu\c{c}\~{a}o do sistema}
\end{figure}

Agora o programa está rodando e o jogador pode acompanhar o que está 
acontecendo, pois o valor do \textit{Input} será exibido constantemente na tela.
Após passar por algum comando o valor de \textit{Input} será modificado de 
acordo com a operação executada.

\begin{figure}[H]
    \includegraphics[width=\textwidth]{../figuras/antes_da_soma.png}
    \caption{Valor do \textit{Input} antes da opera\c{c}\~{a}o}
\end{figure}

Note que após passar pelo comando de soma, o valor de \textit{Input} será 
incrementado em 1, pois foi passado como argumento "input, 1", fazendo com que 
seja somado 1 ao valor corrente do \textit{Input}.

\begin{figure}[H]
    \includegraphics[width=\textwidth]{../figuras/depois_da_soma.png}
    \caption{Valor do \textit{Input} após a opera\c{c}\~{a}o}
\end{figure}

Esta maneira de conseguir acompanhar o que está acontecendo com os valores do 
programa enquanto é executada cada ação permite que o jogador entenda realmente
como cada comando funciona, facilitando o aprendizado principalmente das 
instruções que controlam o fluxo de operação e loops.
