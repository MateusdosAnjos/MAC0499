%!TeX root=../tese.tex
%("dica" para o editor de texto: este arquivo é parte de um documento maior)
% para saber mais: https://tex.stackexchange.com/q/78101/183146
%% ------------------------------------------------------------------------- %%
\chapter{Considerações Finais}
\label{cap:Considerações Finais}

Neste capítulo estarão descritas as dificuldades enfrentadas durante a criação 
do projeto e algumas ideias para futuras implementações.

\section{Usuários e o Tutorial}

Inicialmente o jogo contava com apenas um nível de tutorial, os demais níveis 
eram desafios em que o jogador estaria sozinho. Neste único nível inicial eram
exibidas todas as informações que o jogador precisaria para jogar, desde como 
montar o sistema até a forma de funcionamento dos comandos. Para evitar um 
tutorial demasiado longo e que dificultasse a memorização foi utilizada uma 
linguagem tecnica e concisa.

A primeira experiência com um jogador leigo foi desastrosa. A linguagem técnica 
dificultou muito o entendimento, apenas quem já conhecia programação há algum 
tempo entendia a ideia do jogo. O tamanho do tutorial também dificultou o 
entendimento, pois era passada muita informação de uma só vez.

Houve a primeira refatoração do tutorial, tentando fazer analogia entre a ideia
do jogo e a confecção de um bolo. Os comandos de ajuda de cada comando faziam 
referência ao processo de criação de um bolo, por exemplo, a soma era comparada 
a adicionar ingredientes, os dados de entrada eram os ingredientes iniciais e 
a saída esperada era o sabor de bolo a ser feito.

Apesar de ter melhorado um pouco do entendimento por parte de um leigo, a
quantidade de informação em apenas um nível de tutorial atrapalhava o
entendimento e impossibilitou que a curva de aprendizado fosse satisfatória,
para completar o único nível de tutorial foram necessários aproximadamente 20
minutos, lendo o menu de ajuda e entendendo as mensagens de erro.

Surgiram algumas sugestões de melhorias que eram focadas em modificar as imagens
dos comando, por exemplo, mudar os números de entrada para imagens de 
ingredientes e modificar a saída esperada para a imagem de um bolo. Apesar de 
parecer interessante, já existem diversos jogos com esse estilo e muito 
difícilmente um jogador consegue compreender a relação que existe entre a 
montagem de um sistema que confecciona um bolo utilizando imagens e uma 
sequência de instruções em um código de computador sem ser induzido a isso.

Como \textit{Phoenix Rising} tem o objetivo de ensinar conceitos de programação 
foi decidido manter os comandos o mais próximo do que é utilizado em linguagens
de alto nível e dados de entrada e saída que fossem compatíveis com os que são 
utilizados em tutoriais que introduzem as ideias básicas de programação, 
facilitando a relação entre o jogo e um \textit{script} que automatiza alguma 
tarefa.

A segunda refatoração, portanto, dividiu o tutorial em vários níveis, tentando 
ensinar ao jogador uma ideia de cada vez, ou seja, o primeiro tutorial ensina 
a conectar o sistema, o segundo a mudar as conexões, o terceiro a utilizar o 
primeiro comando e assim por diante. Foi abolida a analogia com o bolo e a 
linguagem, apesar de não ser muito técnica, tentou não se distanciar do que é
utilizado no dia a dia de quem já tem certa experiência com programação, pois 
conhecer os termos técnicos e o modo matemático de se expressar também é 
importante no aprendizado de computação.

Após a segunda refatoração o tutorial foi muito mais efetivo para um leigo
(cada indivíduo leigo jogou o jogo sem nenhuma informação prévia e apenas uma 
vez, portanto o aprendizado foi exclusivamente pela experiência com o 
tutorial), as imagens que ilustram o que é pra ser feito ajudaram bastante
e tornaram \textit{Phoenix Rising} jogável.

\section{Refatorações do Código}

Fazer a refatoração é muito importante para manter a qualidade do código e esse 
processo foi repetido em diversos momentos durante a criação do jogo. Todavia 
essa seção irá tratar de uma refatoração específica que mudou a estrutura do 
projeto e permitiu que a animação mostrada na tela aconteça em tempo de 
execução.

No início da implementação havia apenas os comandos sequenciais, ou seja, 
apenas as operações que não geram desvio. Para tratá-las primeiro era armazenado 
em uma lista todos os comandos do programa criado pelo jogador e, antes da 
animação ocorrer, os dados de entrada eram processados por cada um deles. Em 
outra lista era armazenado o resultado depois da operação de cada comando e 
essa lista era passada para o processo visual exibir na tela.

Da forma que estava o jogador não conseguia ver o momento exato do erro, já que 
todo o programa criado era verificado antes mesmo da animação iniciar, o que
acabava dificultando muito o aprendizado. Ao tentar criar o comando condicional 
a idea de criar uma árvore de execução surgiu, afinal havia a necessidade de 
utilizar uma estrutura de dados que permitisse mapear com facilidade 
as bifurcações.

Criar a árvore de execução envolveu mudanças em vários arquivos, além de 
mudanças nos argumentos recebidos por funções. Entretando essas mudanças 
fizeram com que o comando condicional pudesse existir, abriram possibilidade 
para inserir mais comandos que fazem operações não sequenciais e facilitaram o 
trabalho do processo visual para exibir a animação durante a execução do 
programa.

Mesmo sem entrar em detalhes de como foram feitas essas modificações para 
criar a árvore de execução é válido notar a importância de dispor tempo 
para refatoração e pensar em melhorias para o projeto, pois sem esse 
momento o código ficaria ruim e tornaria muito difícil a 
contribuição de terceiros, além de atrasar o próprio desenvolvimento.

\section{Trabalhos Futuros}

\textit{Phoenix Rising} possui o arcabouço que permite a inserção de várias 
melhorias, como novos comandos, novas conexões e o aperfeiçoamento dos 
aspectos de jogo (\textit{gamification}).

Pelo que já foi explicado em capítulos anteriores a inserção de novas conexões é
trivial, entretanto alguns comandos seriam interessantes, as ideias gerais
sorbe eles são:

\begin{itemize}
    \item[$\bullet$]
        Alocação de memória - Um comando como esse poderia contar com 
        alguma imagem que mostrasse para o jogador qual espaço que foi alocado
        e poderia ser feito com base no comando "variável" trocando
        a variável simples por uma lista.
    \item[$\bullet$]
        Repeat - Um comando de loop poderia ser feito utilizando o filho 
        esquerdo do espaço de ação e um contador de repetições.
    \item[$\bullet$]
        Funções - Criar um comando que permite definir uma função é mais 
        sofisticado, mas permitira ao jogador aprender sobre escopo.
        
        Para criá-lo talvez seja preciso definir um novo comando em 
        \textit{itemDB.gd}, chamado Func, que funcione como apelido para uma 
        sequência de comandos. Então deve-se definir uma forma do jogador criar 
        a sequência de comandos que corresponderá a Func, talvez até uma janela 
        extra na tela de jogo (popup).
\end{itemize}

Estas ideias apenas ilustram o potencial de crescimento do jogo e os possíveis
caminhos para implementar tais ideias mostram que seria factível.

Outro ponto a ser explorado é a \textit{gamificação}. No momento o jogo conta 
apenas com o sistema de pontos simples, porém poderia ser implementado um 
sistema que permitisse ao jogador comprar dicas para cada nível com os pontos 
obtidos, melhorar a função que calcula a pontuação (no momento é uma função
linear) ou até mesmo criar um sistema que permitisse compartilhar soluções entre
amigos em troca de pontos, parecendo um sistema de ajuda entre a comunidade.

O design do jogo também pode ser melhorado, desde a paleta de cores até o 
\textit{leve design},\footnote{Level design é uma parte do desenvolvimento de 
jogos eletrônicos. Envolve a criação de um nível, campanhas e missões.
Level design é também um processo artístico e técnico} já que os quebra cabeças
estão muito simples e os desafios criados não necessitam que o jogador utilize 
muitos comandos para terminá-los.
