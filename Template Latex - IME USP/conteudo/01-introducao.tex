%!TeX root=../tese.tex
%("dica" para o editor de texto: este arquivo é parte de um documento maior)
% para saber mais: https://tex.stackexchange.com/q/78101/183146
%% ------------------------------------------------------------------------- %%
\chapter{Introdução}
\label{cap:introducao}

\section{Motivação e Objetivos}

Com a crescente ascensão da tecnologia nos dias de hoje, o conhecimento sobre
programação tem se tornado cada vez mais importante, não só pelas inúmeras
aplicações que existem, mas também por ser um facilitador, tanto na vida
pessoal quanto na vida profissional.

Devido a esse fato, houve um grande aumento no número de interessados pelo
conhecimento da programação e, consequentemente, o ensino de tal área tem se
difundido cada vez mais. Entretanto, muitos dos interessados por tais técnicas
não dispõem do tempo necessário ou da paciência e concentração para o 
aprendizado tradicional, ou seja, leituras extensas sobre os temas e longas 
sessões práticas para a aplicação das técnicas aprendidas.

Neste momento os jogos ganham força como disseminadores do conhecimento para os 
que buscam o primeiro contato com esta área, pois são
uma forma divertida e rápida de se adquirir experiência básica sobre algo.
Por ser uma forma simples e dinâmica de aprendizado o indíviduo encontra mais
facilidade para encaixar o jogo em sua agenda do que ler um livro teórico sobre
algo.

Desta forma, visando proporcionar um ambiente facilitador do aprendizado dos
conceitos de programação para indivíduos iniciantes ou com pouca experiência,
foi desenvolvido o jogo \textit{Phoenix Rising} que \textit{gamifica}
\footnote{Uso de 
mecânicas e dinâmicas de jogos para engajar pessoas, resolver problemas e 
melhorar o aprendizado, motivando ações e comportamentos em ambientes fora do 
contexto de jogos.} uma plataforma de ensino. Além disso a estrutura do 
código foi pensada de modo a facilitar a inserção de novas características ao 
jogo pelos indivíduos que têm certa experiência em programação, fazendo com que
o projeto desenvolvido sirva para uma grande parte dos interessados em 
aprofundar o conhecimento.


%% ------------------------------------------------------------------------- %%
\section{Organização do Projeto}
\label{sec:consideracoes_preliminares}

O projeto foi desenvolvido utilizando Godot na versão 3.1.1 stable, 
uma \textit{game engine} que facilita a produção de jogos e possui uma linguagem
própria chamada GDScript.
Todo o código do jogo está mantido no GitHub, portanto o projeto é open source,
o que facilita a contribuição pela comunidade.

Como um dos objetivos do projeto é disponibilizar o código fonte para
melhorias serem implementadas, o código e comentários estão em inglês, seguindo
as boas práticas de programação. Vale salientar também que a eficiência não foi
o principal ponto do projeto mas sim a legibilidade e a flexibilidade do 
código, portanto em algumas partes preferiu-se utilizar um pouco mais de memória
e/ou processamento, embora tais escolhas não tenham grande impacto na 
jogabilidade.

%% ------------------------------------------------------------------------- %%
\section{Visão Geral do Jogo}
\label{sec:consideracoes_preliminares}

Para nortear o leitor neste trabalho, aqui está uma breve descrição sobre como o
jogo funciona.

O objetivo do jogador é completar o maior número de desafios no menor tempo
possível, maximizando o somatório de pontos. Para isso ele deve resolver o 
problema de cada nível criando um programa que processa os dados de entrada e 
devolve dados de saída iguais ao dados de saída esperados no nível.

A resolução do desafio pode ser dividido em duas etapas, a primeira é resolver 
o quebra cabeças e a segunda é posicionar os comandos de programação da forma 
correta, para que o programa processe corretamente os dados. O quebra cabeças 
consiste em conectar a entrada dos dados com a saída da resposta esperada, 
utilizando conexões específicas. 
Após completado o primeiro desafio, o jogador deve posicionar os comandos
disponibilizados para criar o programa que solucionará o problema daquele nível.

Tudo isso deve ser feito no menor tempo possível, pois há um cronômetro que 
marca quanto tempo o jogador tem para solucionar o nível e os pontos ganhos são 
diretamente porporcionais ao tempo restante.

Para auxiliar o processo de aprendizado o jogador conta com mensagens de guia no 
início do jogo além de uma animação que aparece na tela ao executar o programa 
criado. Nesta animação é possível ver o que acontece com os valores de entrada 
e até mesmo acompanhar como as variáveis do programa estão mudando.

As informações sobre o código do jogo estarão explicadas mais adiante. As 
explicações incluem como os arquivos estão organizados, como cenas mais 
importantes foram implementadas e algumas imagens que exemplificam trechos de 
alguns \textit{scripts}\footnote{conjunto de instruções para que uma função 
(ou método) seja executada em determinado aplicativo.} mais importantes.
