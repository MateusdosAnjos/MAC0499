%!TeX root=../tese.tex
%("dica" para o editor de texto: este arquivo é parte de um documento maior)
% para saber mais: https://tex.stackexchange.com/q/78101/183146
%% ------------------------------------------------------------------------- %%
\chapter{Conclusão}
\label{cap:Conclusão}

No ponto atual, \textit{Phoenix Rising} conseguiu unir uma plataforma de 
ensino com elementos de jogo, tornando-se mais atrativo para pessoas que
queiram encaixar o aprendizado da programação no dia a dia.

O jogo não só auxilia iniciantes em computação a entender melhor o que 
está acontecendo com a entrada a cada comando, facilitando o aprendizado 
nos primeiros meses, como também é uma boa forma do estudante intermediário 
se familiarizar com certas estruturas que foram utilizadas no código, por 
exemplo, árvores, listas, dicionários, bem como alguns algoritmos como busca
em largura em uma árvore.

É importante ressaltar também que a continuação do projeto foi facilitada, já 
que o código foi pensado com esse propósito. Sendo assim é possível afirmar que
o jogo \textit{Phoenix Rising} alcançou o objetivo para o qual foi
desenvolvido.