%!TeX root=../tese.tex
%("dica" para o editor de texto: este arquivo é parte de um documento maior)
% para saber mais: https://tex.stackexchange.com/q/78101/183146

% O resumo é obrigatório, em português e inglês. Este comando também gera
% automaticamente a referência para o próprio documento, conforme as normas
% sugeridas da USP
\begin{resumo}{port}

    A ideia do trabalho é desenvolver um jogo que facilite o aprendizado de 
    conceitos iniciais de programação e que, ao mesmo tempo, tenha um código 
    facilmente personalizável, fazendo com que os indivíduos tenham
    interesse em jogar e alterar as características do jogo modificando o código
    fonte, tornando o projeto um ambiente mais completo de aprendizado. 

    Para atender a demanda do projeto foi desenvolvido \textit{Phoenix Rising}, 
    um jogo cujo objetivo de cada nível é criar um
    programa que processa dados de entrada e devolve uma determinada
    resposta, conquistando pontos ao concluir cada desafio. Para isso 
    o jogador deve solucionar um tipo de quebra cabeças e utilizar 
    conceitos de programação.
    
    A fim de facilitar o aprendizado, existe uma animação
    que aparece na tela e mostra como os valores do programa criado se
    modificam a cada comando executado.
    
    Já para os jogadores que gostam de efetuar algumas modificações e
    criar novos comandos, por exemplo, o código é aberto e foi pensado 
    de maneira a facilitar as personalizações e a continuação do projeto.

\end{resumo}
    
    % O resumo é obrigatório, em português e inglês. Este comando também gera
    % automaticamente a referência para o próprio documento, conforme as normas
    % sugeridas da USP
    % \begin{resumo}{eng}
    % Elemento obrigatório, elaborado com as mesmas características do resumo em
    % língua portuguesa. De acordo com o Regimento da Pós-Graduação da USP (Artigo
    % 99), deve ser redigido em inglês para fins de divulgação. É uma boa ideia usar
    % o sítio \url{www.grammarly.com} na preparação de textos em inglês.
    % Text text text text text text text text text text text text text text text text
    % text text text text text text text text text text text text text text text text
    % text text text text text text text text text text text text text text text text
    % text text text text text text text text text text text text.
    % Text text text text text text text text text text text text text text text text
    % text text text text text text text text text text text text text text text text
    % text text text.
    % \end{resumo}
    
    