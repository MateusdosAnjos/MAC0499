%!TeX root=../tese.tex
%("dica" para o editor de texto: este arquivo é parte de um documento maior)
% para saber mais: https://tex.stackexchange.com/q/78101/183146

%% ------------------------------------------------------------------------- %%
\chapter{Introdução}
\label{cap:introducao}

\section{Motivação e Objetivos}

Com a crescente ascensão da tecnologia nos dias de hoje o conhecimento sobre
programação tem se tornado cada vez mais importante, não só pelas inúmeras
aplicações que existem, mas também por ser um facilitador, tanto na vida
pessoal quanto na vida profissional.

Devido a esse fato, houve um grande aumento no número de interessados pelo
conhecimento da programação e, consequentemente, o ensino de tal área tem se
difundido cada vez mais. Entretanto, muitos dos interessados por tais técnicas
não dispõem do tempo necessário ou da paciência e concentração para o aprendizado
tradicional, ou seja, leituras extensas sobre os temas e longas sessões práticas
para a aplicação das técnicas aprendidas.

Neste momento os jogos ganham força como disseminadores do conhecimento para os 
que buscam o primeiro contato com esta área, pois são
uma forma divertida e rápida de se adquirir experiência básica sobre algo.
Por ser uma forma simples e dinâmica de aprendizado o indíviduo encontra mais
facilidade para encaixar o jogo em sua agenda do que ler um livro teórico sobre
algo. Por isso que o jogo desenvolvido tenta \textit{gamificar} uma plataforma de 
ensino.

Desta forma, visando proporcionar um ambiente facilitador do aprendizado dos
conceitos de programação para indivíduos iniciantes ou com pouca experiência
foi desenvolvido o jogo Phoenix Rising. Além disso a estrutura do código foi
pensada de modo a facilitar a inserção de novas características ao jogo
pelos indivíduos que têm certa experiência em programação, fazendo com que
o projeto desenvolvido sirva para uma grande parte dos interessados em 
aprofundar o conhecimento.


%% ------------------------------------------------------------------------- %%
\section{Organização do Projeto}
\label{sec:consideracoes_preliminares}

O projeto foi desenvolvido utilizando Godot na versão 3.1.1 stable, 
uma \textit{game engine} que facilita a produção de jogos e possui uma linguagem
própria chamada GDScript.
Todo o código do jogo está mantido no GitHub, portanto o projeto é open source,
o que facilita a contribuição pela comunidade.

Como um dos objetivos do projeto é disponibilizar o código fonte para
melhorias serem implementadas, o código e comentários estão em inglês, seguindo
as boas práticas de programação. Vale salientar também que a eficiência não foi
principal ponto do projeto mas sim a legibilidade e a flexibilidade do 
código, portanto em algumas partes preferiu-se utilizar um pouco mais de memória
e/ou processamento, embora tais escolhas não tenham grande impacto na jogabilidade.

% \section{Ferramentas Bibliográficas}

% Embora seja possível pesquisar por material acadêmico na Internet usando sistemas
% de busca ``comuns'', existem ferramentas dedicadas, como o \textsf{Google Scholar}\index{Google Scholar}
% (\url{scholar.google.com}). Você também pode querer usar o \textsf{Web of Science}\index{Web of Science}
% (\url{webofscience.com}) e o \textsf{Scopus}\index{Scopus} (\url{scopus.com}), que oferecem
% recursos sofisticados e limitam a busca a periódicos com boa reputação acadêmica.
% Essas duas plataformas não são gratuitas, mas os alunos da USP têm acesso a elas
% através da instituição. Ambas são capazes de exportar os dados para o formato .bib,
% usado pelo \LaTeX{}. Algumas editoras, como a ACM e a IEEE, também têm sistemas de
% busca bibliográfica.

% Apenas uma parte dos artigos acadêmicos de interesse está disponível livremente
% na Internet; os demais são restritos a assinantes. A CAPES assina um grande
% volume de publicações e disponibiliza o acesso a elas para diversas universidades
% brasileiras, entre elas a USP, através do seu portal de periódicos
% (\url{periodicos.capes.gov.br}). Existe uma extensão para os navegadores
% Chrome e Firefox (\url{www.infis.ufu.br/capes-periodicos}) que facilita o uso
% cotidiano do portal.

% Para manter um banco de dados organizado sobre artigos e outras fontes bibliográficas
% relevantes para sua pesquisa, é altamente recomendável que você use uma ferramenta
% como Zotero~(\url{zotero.org})\index{Zotero} ou
% Mendeley~(\url{mendeley.com})\index{Mendeley}. Ambas podem exportar seus dados no
% formato .bib, compatível com \LaTeX{}. Também existem três plataformas
% gratuitas que permitem a busca de referências acadêmicas já no formato .bib:

% \begin{itemize}
%   \item \emph{CiteULike}\index{CiteULike} (patrocinados por Springer): \url{www.citeulike.org}
%   \item Coleção de bibliografia em Ciência da Computação: \url{liinwww.ira.uka.de/bibliography}
%   \item Google acadêmico\index{Google Scholar} (habilitar bibtex nas preferências): \url{scholar.google.com}
% \end{itemize}

% Lamentavelmente, ainda não existe um mecanismo de verificação ou validação das
% informações nessas plataformas. Portanto, é fortemente sugerido validar todas
% as informações de tal forma que as entradas bib estejam corretas.

% De qualquer modo, tome muito cuidado na padronização das referências
% bibliográficas: ou considere TODOS os nomes dos autores por extenso, ou TODOS
% os nomes dos autores abreviados.  Evite misturas inapropriadas.

% \section{O Que o IME Espera}

% Ao terminar sua tese/dissertação, você deve entregar uma cópia dela para a
% CPG. Após a defesa, você tem 30 dias para revisar o texto e incorporar as
% sugestões da banca. Assim, há duas versões oficiais do documento: a versão
% original e a versão corrigida, o que deve ser indicado na folha de rosto.
% \index{Tese/Dissertação!versões}

% Fica a critério do aluno definir aspectos como o tamanho de fonte, margens,
% espaçamento, estilo de referências, cabeçalho, etc. considerando sempre o
% bom senso. A CPG, em reunião realizada em junho de 2007, aprovou que as
% teses/dissertações deverão seguir o formato padrão por ela
% definido\footnote{\url{www.ime.usp.br/dcc/pos/normas/tesesedissertacoes}}.
% Esse padrão refere-se aos itens que devem estar presentes nas teses/dissertações
% (e.g. capa, formato de rosto, sumário, etc.), e não à formatação do documento.
% Ele define itens obrigatórios e opcionais, conforme segue:\index{Formatação}
% \index{Tese/Dissertação!itens obrigatórios}
% \index{Tese/Dissertação!itens opcionais}

% \begin{itemize}
%   \item \textsc{Capa} (obrigatória)
%   \begin{itemize}
%     \item O IME usa uma capa padrão de cartolina para todas as
%     teses/dissertações.  Essa capa tem uma janela recortada por onde se
%     vê o título e o autor do trabalho e, portanto, a capa impressa do
%     trabalho deve incluir o título e o autor na posição correspondente da
%     página. Ela fica centralizada na página, tem 100mm de largura, 60mm de
%     altura e começa 47mm abaixo do topo da página.

%     \item O título da tese/dissertação deverá começar com letra maiúscula
%     e o resto deverá ser em minúsculas, salvo nomes próprios.

%     \item O nome do aluno(a) deverá ser completo e sem abreviaturas.

%     \item É preciso explicitar se é uma tese ou dissertação (para
%     obtenção do título de doutor, tese; para obtenção do título de
%     mestre, dissertação).

%     \item O nome do programa deve constar da capa (Matemática,
%     Matemática Aplicada, Estatística ou Ciência da Computação).

%     \item Também devem constar o nome completo do orientador e do
%     co-orientador, se houver.

%     \item Se o aluno recebeu bolsa, deve-se indicar a(s) agência(s).

%     \item É preciso informar o mês e ano do depósito ou da entrega da
%     versão corrigida.
%   \end{itemize}

%   \item \textsc{Folha de Rosto} (obrigatória, tanto para a versão
%   depositada quanto para a versão corrigida)
%   \begin{itemize}
%     \item o título da tese/dissertação deverá seguir o padrão da capa

%     \item deve informar se se trata da versão original ou da versão
%     corrigida; no segundo caso, deve também incluir os nomes
%     dos membros da banca.
%   \end{itemize}

%   \item \textsc{Agradecimentos} (opcional)

%   \item \textsc{Resumo}, em português (obrigatório)

%   \item \textsc{Abstract}, em inglês (obrigatório)

%   \item \textsc{Sumário} (obrigatório)

%   \item \textsc{Listas} (opcionais)
%   \begin{itemize}
%     \item Lista de Abreviaturas
%     \item Lista de Símbolos
%     \item Lista de Figuras
%     \item Lista de Tabelas
%   \end{itemize}

%   \item \textsc{Referências Bibliográficas} (obrigatório)

%   \item \textsc{Índice Remissivo} (opcional\footnote{O índice remissivo
%    pode ser muito útil para a banca; assim, embora seja um item opcional,
%    recomendamos que você o crie.})
%\end{itemize}
