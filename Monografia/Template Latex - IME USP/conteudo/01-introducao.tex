%!TeX root=../tese.tex
%("dica" para o editor de texto: este arquivo é parte de um documento maior)
% para saber mais: https://tex.stackexchange.com/q/78101/183146

%% ------------------------------------------------------------------------- %%
\chapter{Introdução}
\label{cap:introducao}

\section{Motivação e Objetivos}

Com a crescente ascensão da tecnologia nos dias de hoje o conhecimento sobre
programação tem se tornado cada vez mais importante, não só pelas inúmeras
aplicações que existem, mas também por ser um facilitador, tanto na vida
pessoal quanto na vida profissional.

Devido a esse fato, houve um grande aumento no número de interessados pelo
conhecimento da programação e, consequentemente, o ensino de tal área tem se
difundido cada vez mais. Entretanto, muitos dos interessados por tais técnicas
não dispõem do tempo necessário ou da paciência e concentração para o 
aprendizado tradicional, ou seja, leituras extensas sobre os temas e longas 
sessões práticas para a aplicação das técnicas aprendidas.

Neste momento os jogos ganham força como disseminadores do conhecimento para os 
que buscam o primeiro contato com esta área, pois são
uma forma divertida e rápida de se adquirir experiência básica sobre algo.
Por ser uma forma simples e dinâmica de aprendizado o indíviduo encontra mais
facilidade para encaixar o jogo em sua agenda do que ler um livro teórico sobre
algo. Por isso que o jogo desenvolvido tenta \textit{gamificar} uma plataforma 
de ensino.

Desta forma, visando proporcionar um ambiente facilitador do aprendizado dos
conceitos de programação para indivíduos iniciantes ou com pouca experiência
foi desenvolvido o jogo Phoenix Rising. Além disso a estrutura do código foi
pensada de modo a facilitar a inserção de novas características ao jogo
pelos indivíduos que têm certa experiência em programação, fazendo com que
o projeto desenvolvido sirva para uma grande parte dos interessados em 
aprofundar o conhecimento.


%% ------------------------------------------------------------------------- %%
\section{Organização do Projeto}
\label{sec:consideracoes_preliminares}

O projeto foi desenvolvido utilizando Godot na versão 3.1.1 stable, 
uma \textit{game engine} que facilita a produção de jogos e possui uma linguagem
própria chamada GDScript.
Todo o código do jogo está mantido no GitHub, portanto o projeto é open source,
o que facilita a contribuição pela comunidade.

Como um dos objetivos do projeto é disponibilizar o código fonte para
melhorias serem implementadas, o código e comentários estão em inglês, seguindo
as boas práticas de programação. Vale salientar também que a eficiência não foi
principal ponto do projeto mas sim a legibilidade e a flexibilidade do 
código, portanto em algumas partes preferiu-se utilizar um pouco mais de memória
e/ou processamento, embora tais escolhas não tenham grande impacto na 
jogabilidade.

\chapter{Conceitos Básicos}
\label{cap:Conceitos Básicos}

\section{O Conceito de Árvore}

Para facilitar o entendimento, deve-se entender um pouco sobre o que
é uma árvore no escopo da programação, pois tal conceito aparecerá muitas
vezes neste trabalho, entretanto a definição informal, passando apenas a ideia 
do funcionamento, bastará para entender este projeto.

Árvore refere-se a uma forma de estruturar os dados de um programa,
informalmente pode ser definido como um conjunto de elementos que armazenam 
informações, por sua vez são os chamados nós. Toda árvore possui o elemento 
chamado raiz, que é primeiro nó, de onde a árvore começa, e que possui ligações
para outros elementos denominados filhos, por sua vez também são nós.

\begin{figure}[h]
    \includegraphics[width=\linewidth]{../figuras/arvore.png}
    \caption{Exemplo de árvore}
\end{figure}   

Perceba que a árvore cresce para baixo, sendo que a raiz dá origem a tudo.
Os nós A, B, C, D são filhos da Raiz. Os nós E, F são filhos do nó A. O nó
D é filho da Raiz e não tem filhos.

Como a estrutura dos projetos criados utilizando a \textit{Godot Engine} é 
baseada em árvores, já é possível entender parte de como o jogo desenvolvido foi 
estruturado. Entretanto ainda é necessário explicar o que é uma 
\textit{Game Engine}.


\section{O que é \textit{Game Engine?}}

Uma \textit{game engine} é um programa para computador com um conjunto de 
bibliotecas capaz de juntar e construir, em tempo real, todos os elementos de um
jogo.
Ela inclui motor gráfico para renderizar gráficos em 2D ou 3D, motor de física 
para detectar colisões e fazer animações, além de suporte para sons, 
inteligência artificial, gerenciamento de arquivos, programação, entre outros.
Por conta dessas facilidades, a partir do uso de uma \textit{game engine}, é 
possível criar um jogo do zero de maneira mais simples e replicar vários estilos
jogos com mais facilidade.

Escolher a Godot para este projeto teve como motivação o grupo de extensão 
USPGameDev, além do aprendizado ser relativamente simples e de ser um 
\textit{software open source} sob a licença MIT, desenvolvido de forma 
independente pela comunidade.

Como foi estabelecido o conhecimento sobre alguns termos gerais, agora é
possível entender o básico de como funciona a \textit{Godot Engine}.

\section{Entendendo sobre a \textit{Godot Engine}}

A seguir estão as explicações dos conceitos básicos.

\subsection{Nós}

Nós são blocos de construção fundamentais para a criação de um jogo. Um nó pode
executar uma variedade de funções especializadas.
No entanto, qualquer nó fornecido sempre possui os seguintes atributos:
\begin{itemize}
    \item[$\bullet$]
        Possui um nome.
    \item[$\bullet$]
        Possui propriedades editáveis.
    \item[$\bullet$]
        Ele pode receber um retorno de chamada (\textit{callback}) para 
        processar todos os quadros (\textit{frames}).
    \item[$\bullet$]
        Pode ser estendido (para ter mais funções).
    \item[$\bullet$]
        Pode ser adicionado a outro nó como filho.
\end{itemize}

Perceba que o último atributo é muito importante, pois quando nós tem outros nós
como filhos o conjunto se torna uma árvore, como foi explicado anteriormente.
Em Godot, a capacidade de organizar nós dessa maneira cria uma ferramenta 
poderosa para organizar projetos. Como nós diferentes têm funções diferentes, 
combiná-los permite a criação de funções mais complexas, a partir disso
\textit{Phoenix Rising} foi criado.

\subsection{Cenas}

Uma cena é composta por um grupo de nós organizados hierarquicamente 
(em forma de árvore). Além disso, uma cena:

\begin{itemize}
    \item[$\bullet$]
        Sempre tem um nó raiz.
    \item[$\bullet$]
        Pode ser salvo no disco e carregado de volta.
    \item[$\bullet$]
        Pode ser instanciado (Explicado adiante).
\end{itemize}

Executar um jogo significa executar uma cena. Um projeto pode conter várias 
cenas, mas para o jogo começar, uma delas deve ser selecionada como a cena 
principal.

Basicamente, o editor Godot é um editor de cenas. Possui muitas ferramentas para
editar cenas 2D e 3D, bem como interfaces com o usuário, mas o editor é baseado 
no conceito de edição de uma cena e nos nós que a compõem.

\subsection{Instâncias}

Criar uma única cena e adicionar nós a ela pode funcionar para pequenos 
projetos, mas à medida que o projeto aumenta em tamanho e complexidade, o número
de nós pode se tornar rapidamente incontrolável. Para resolver isso, Godot 
permite que um projeto seja separado em qualquer número de cenas. Isso fornece 
uma ferramenta poderosa que ajuda a organizar os diferentes componentes do seu
jogo.

Em Cenas e nós, você aprendeu que uma cena é uma coleção de nós organizados em 
uma estrutura de árvore, com um único nó como raiz da árvore.
Você pode criar quantas cenas quiser e salvá-las em disco. As cenas salvas dessa
maneira são chamadas de "Cenas compactadas" \ (\textit{packed scenes}) e têm uma 
extensão de nome de arquivo ".tscn".

A instanciação é muito utilizada em \textit{Phoenix Rising}, portanto esta
parte deve ficar mais clara conforme adentramos nos detalhes da estrutura e
de implementação mais adiante.

\subsection{\textit{SceneTree}}

Para entender melhor o que é uma \textit{SceneTree} deve-se entender um pouco 
sobre o modo como \textit{Godot} trabalha internamente.

Primeiro, a única instância que é executada no 
início pertence à classe \textit{OS}. Depois, todos os drivers, servidores,
linguagens de script, sistema de cenas e outros recursos são carregados.

Quando a inicialização estiver concluída, o sistema operacional precisará 
receber um \textit{MainLoop} para executar. Até o momento, tudo isso funciona 
internamente (você pode verificar o arquivo "main.cpp" no código-fonte se 
estiver interessado em ver como isso funciona internamente).

Este \textit{MainLoop} da inicio ao programa do usuário, ou jogo. Essa classe
possui alguns métodos, para inicialização, \textit{callbacks} e \textit{input}. 
Novamente, esse é um nível baixo e, ao fazer jogos em Godot, escrever seu 
próprio \textit{MainLoop} raramente faz sentido.

A partir disso o sistema de cena fornece seu próprio loop principal para o 
\textit{OS}, chamado de \textit{SceneTree}. 
Isso é instanciado automaticamente e definido ao executar uma cena, sem a 
necessidade de fazer nenhum trabalho extra.

Agora que a \textit{SceneTree} foi introduzida é importante saber que ela existe
e possui algumas características, como:

\begin{itemize}
    \item[$\bullet$]
        Contém o \textit{Viewport} raiz, ao qual uma cena é adicionada 
        como filha quando é aberta pela primeira vez para se tornar parte da
        \textit{SceneTree}.
    \item[$\bullet$]
        Contém informações sobre os grupos e possui os meios para chamar 
        todos os nós em um grupo ou obter uma lista deles.
    \item[$\bullet$]
        Contém algumas funcionalidades do estado atual do
        jogo, como definir o modo de pausa ou término de processos.
\end{itemize}

Desta forma, quando um nó é conectado, direta ou indiretamente, à 
\textit{viewport} raiz, ele se torna parte da \textit{SceneTree}. Quando os nós 
entram na Árvore da cena, eles se tornam ativos. Eles têm acesso a tudo o que 
precisam para processar, obter informações, exibir imagens em 2D e 3D, receber 
e enviar notificações, reproduzir sons, entre outros processamentos. Quando são
removidos da árvore da cena, perdem essas habilidades, evitado alguns 
comportamentos indesejados.

A importância de se entender tudo isso, para este projeto, se dá pois a maioria
das operações de nó em \textit{Godot}, como desenhar 2D, processar ou obter 
notificações, são feitas seguindo a ordem que os nós estão na árvore.

O processo de tornar um nó ativo ao entrar na \textit{SceneTree} se dá seguindo
os passos:
\begin{itemize}
    \item[1.]
        Uma cena é carregada do disco ou criada por script.
    \item[2.]
        A raiz dessa cena é adicionada como filha de \textit{Viewport}, ou como 
        filha de qualquer filha de \textit{Viewport}
    \item[3.]
        Cada nó da cena recém-adicionada receberá a notificação 
        \textit{"enter\_tree"} na ordem de cima para baixo, ou seja, o pai é
        notificado e depois cada um de seus filhos.
    \item[4.]
        Uma notificação extra, \textit{"ready"} é fornecida por conveniência, 
        quando um nó e todos os seus filhos estão dentro da cena ativa.
    \item[5.]
        Quando uma cena (ou parte dela) é removida, eles recebem a notificação
        \textit{"exit\_tree"} na ordem de baixo para cima, ou seja, os filhos são
        notificados e depois o pai.
\end{itemize}

\subsection{Singleton}

O sistema de cenas utilizado em \textit{Godot}, embora poderoso e flexível, tem 
uma desvantagem: não há método para armazenar informações, por exemplo, 
pontuação do jogador (inclusive utilizado neste projeto), que é necessário para 
mais de uma cena.

Existem alternativas de implementação ao se deparar com estes problemas, porém
na maioria dos casos o padrão \textit{Singleton} irá consumir menos tempo e
memória. Isso deve-se ao fato de \textit{Singleton} ser uma ferramenta útil para
resolver o caso de uso comum em que você precisa armazenar informações 
persistentes entre as cenas. No nosso caso, é possível reutilizar a mesma cena 
ou classe para vários \textit{Singletons}, desde que eles tenham 
nomes diferentes.

Resumindo, usando esse conceito, você pode criar objetos que:

\begin{itemize}
    \item[$\bullet$]
        Sempre estejam carregados e prontos para uso, independentemente da cena 
        em execução no momento.
    \item[$\bullet$]
        Pode armazenar variáveis globais, como informações do jogador.
    \item[$\bullet$]
        Pode lidar com alternância de cenas e transições entre cenas.
\end{itemize}

Vale lembrar também que o carregamento automático de nós e scripts pode nos dar 
essas características ao custo de processamento.

\subsection{Sinais}

Sinais permitem que um nó envie uma mensagem que outros nós possam ouvir e 
responder. Por exemplo, em vez de verificar continuamente um botão para ver se 
ele está sendo pressionado, o botão pode emitir um sinal quando é pressionado e
assim quem receber o sinal poderá executar o que é necessário.

Servem, portanto, para dissociar os objetos do jogo, o que leva a um código 
melhor organizado, mais legível e limpo. Também faz com que os objetos do jogo
não precisem estar sempre em conexão com outros, pois um nó pode emitir um sinal
e apenas os nós interessados em tratar tal evento, aqueles que o emissor se
conectou, recebam este sinal.

Alguns nós já vem com uma serie de sinais prontos para serem conectados, como
visto na figura abaixo:

\begin{figure}[H]
    \includegraphics[scale=0.4]{../figuras/sinais_pre_programados.png}
    \caption{Sinais pré programados de um nó do tipo \textit{Button}}
\end{figure}

Note que o sinal \textit{pressed()} relativo ao nó chamado \textit{FullScreen} 
já está conectado.

Entretando nem sempre estes sinais cobrem a necessidade do projeto. Sendo assim
é preciso criar o próprio sinal, utilizando código. Veja o exemplo abaixo:

%\begin{figure}[h]
%    \includegraphics[width=\linewidth]{../figuras/arvore.png}
%    \caption{Sinal criado por código}
%\end{figure}

Depois pode-se conectar o sinal utilizando a interface de programação
que o \textit{Godot} oferece ou conectá-lo via código como visto na imagem
a seguir:

%\begin{figure}[h]
%    \includegraphics[width=\linewidth]{../figuras/arvore.png}
%    \caption{Sinal conectado por código}
%\end{figure}


\subsection{\textit{GDScripts}}

\textit{GDScript} é uma linguagem de programação de alto nível e tipagem 
dinâmica usada para criar e modelar o comportamento dos nós.
Ela usa uma sintaxe semelhante ao \textit{Python} (os blocos são 
baseados em identação e muitas palavras-chave são semelhantes). 
Seu objetivo é ser otimizada e fortemente integrada ao Godot Engine, permitindo
grande flexibilidade para criação e integração de conteúdo.

Quando adicionado ao nó o script adiciona comportamento a ele, controlando seu
funcionamento e as interações com outros nós: filhos, pais, irmãos e assim por 
diante. O escopo local do script é o próprio nó. Em outras palavras, o script 
herda as funções fornecidas por esse nó.

\section{Bibliografia}

\textsc{Referências Bibliográficas} 

\textnormal{https://www.cse.unr.edu/~sushil/class/gas/papers/GameAIp27-lewis.pdf}
%https://docs.godotengine.org/en/3.1/getting_started/step_by_step/scenes_and_nodes.html
%https://docs.godotengine.org/en/3.1/getting_started/step_by_step/instancing.html
%https://docs.godotengine.org/en/3.1/getting_started/scripting/gdscript/gdscript_basics.html#doc-gdscript
%https://docs.godotengine.org/en/latest/getting_started/step_by_step/scene_tree.html
%https://docs.godotengine.org/en/3.1/getting_started/step_by_step/signals.html