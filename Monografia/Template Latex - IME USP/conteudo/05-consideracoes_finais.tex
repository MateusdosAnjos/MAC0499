%!TeX root=../tese.tex
%("dica" para o editor de texto: este arquivo é parte de um documento maior)
% para saber mais: https://tex.stackexchange.com/q/78101/183146
%% ------------------------------------------------------------------------- %%
\chapter{Considerações Finais}
\label{cap:Considerações Finais}

Neste capítulo estarão descritas as dificuldades enfrentadas durante a criação 
do projeto e algumas ideias para futuras implementações.

\section{Usuários e o Tutorial}

Inicialmente o jogo contava com apenas um nível de tutorial, os demais níveis 
eram desafios em que o jogador estaria sozinho. Neste único nível inicial eram
exibidas todas as informações que o jogador precisaria para jogar, desde como 
montar o sistema até a forma de funcionamento dos comandos. Para evitar um 
tutorial demasiado longo e que dificultasse a memorização foi utilizada uma 
linguagem tecnica e concisa.

A primeira experiência com um jogador leigo foi desastrosa. A linguagem técnica 
dificultou muito o entendimento, apenas quem já conhecia programação há algum 
tempo entendia a ideia do jogo. O tamanho do tutorial também dificultou o 
entendimento, pois era passada muita informação de uma só vez.

Houve a primeira refatoração do tutorial, tentando fazer analogia entre a ideia
do jogo e a confecção de um bolo. Os comandos de ajuda de cada comando faziam 
referência ao processo de criação de um bolo, por exemplo, a soma era comparada 
a adicionar ingredientes, os dados de entrada eram os ingredientes iniciais e 
a saída esperada era o sabor de bolo a ser feito.

Apesar de ter melhorado um pouco do entendimento por parte de um leigo, a
quantidade de informação em apenas um nível de tutorial atrapalhava o
entendimento e impossibilitou que a curva de aprendizado fosse satisfatória,
para completar o único nível de tutorial foram necessários aproximadamente 20
minutos, lendo o menu de ajuda e entendendo as mensagens de erro.

Surgiram algumas sugestões de melhorias que eram focadas em modificar as imagens
dos comando, por exemplo, mudar os números de entrada para imagens de 
ingredientes e modificar a saída esperada para a imagem de um bolo. Apesar de 
parecer interessante, já existem diversos jogos com esse estilo e muito 
difícilmente um jogador consegue compreender a relação que existe entre a 
montagem de um sistema que confecciona um bolo utilizando imagens e uma 
sequência de instruções em um código de computador sem ser induzido a isso.

Como \textit{Phoenix Rising} tem o objetivo de ensinar conceitos de programação 
foi decidido manter os comandos o mais próximo do que é utilizado em linguagens
de alto nível e dados de entrada e saída que fossem compatíveis com os que são 
utilizados em tutoriais que introduzem as ideias básicas de programação, 
facilitando a relação entre o jogo e um \textit{script} que automatiza alguma 
tarefa.

A segunda refatoração, portanto, dividiu o tutorial em vários níveis, tentando 
ensinar ao jogador uma ideia de cada vez, ou seja, o primeiro tutorial ensina 
a conectar o sistema, o segundo a mudar as conexões, o terceiro a utilizar o 
primeiro comando e assim por diante. Foi abolida a analogia com o bolo e a 
linguagem, apesar de não ser muito técnica, tentou não se distanciar do que é
utilizado no dia a dia de quem já tem certa experiência com programação, pois 
conhecer os termos técnicos e o modo matemático de se expressar também é 
importante no aprendizado de computação.

Após a segunda refatoração o tutorial foi muito mais efetivo para um leigo
(cada indivíduo leigo jogou o jogo sem nenhuma informação prévia e apenas uma 
vez, portanto o aprendizado foi exclusivamente pela experiência com o 
tutorial), as imagens que ilustram o que é pra ser feito ajudaram bastante
e tornaram \textit{Phoenix Rising} jogável.

\section{Refatorações do Código}

\section{Trabalhos Futuros}

