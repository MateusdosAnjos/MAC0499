%!TeX root=../tese.tex
%("dica" para o editor de texto: este arquivo é parte de um documento maior)
% para saber mais: https://tex.stackexchange.com/q/78101/183146
%% ------------------------------------------------------------------------- %%
\chapter{Considerações Finais}
\label{cap:Considerações Finais}

Neste capítulo estarão descritas as dificuldades enfrentadas durante a criação 
do projeto e algumas ideias para futuras implementações.

\section{Usuários e o Tutorial}

Inicialmente o jogo contava com apenas um nível de tutorial, os demais níveis 
eram desafios em que o jogador estaria sozinho. Neste único nível inicial eram
exibidas todas as informações que o jogador precisaria para jogar, desde como 
montar o sistema até a forma de funcionamento dos comandos. Para evitar um 
tutorial demasiado longo e que dificultasse a memorização foi utilizada uma 
linguagem tecnica e concisa.

A primeira experiência com um jogador leigo foi desastrosa. A linguagem tecnica 
dificultou muito o entendimento, apenas quem já conhecia programação há algum 
tempo entendia a ideia do jogo. O tamanho do tutorial também dificultou o 
entendimento, pois era passada muita informação de uma só vez.

\section{Refatorações do Código}

\section{Trabalhos Futuros}

